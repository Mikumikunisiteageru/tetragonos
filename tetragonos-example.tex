% arara: clean: {files: [tetragonos-example.glsdefs, tetragonos-example.TG-*]}
% arara: xelatex
% arara: makeglossaries
% arara: xelatex
% arara: xelatex

%% tetragonos-example.sty
%% Copyright 2019 Yuchang Yang < yang.yc.allium@gmail.com >
%
% This work may be distributed and/or modified under the
% conditions of the LaTeX Project Public License, either version 1.3
% of this license or (at your option) any later version.
% The latest version of this license is in
%   http://www.latex-project.org/lppl.txt
% and version 1.3 or later is part of all distributions of LaTeX
% version 2005/12/01 or later.
%
% This work has the LPPL maintenance status `maintained'.
% 
% The Current Maintainer of this work is Yuchang Yang.
%
% This work consists of the files tetragonos.sty and tetragonos-database.tex
% and the associated example file tetragonos-example.tex

\documentclass[a5paper]{ctexart}
\usepackage[margin=25mm]{geometry}
\usepackage{multicol}
\usepackage{tetragonos}
\usepackage[nopostdot,nomain]{glossaries}
\newglossary*{TG}{Tegragonos}
\makeglossaries
\def\syntaxTG#1#2#3#4#5{\textbf{#1#2#3#4\textsubscript{#5}}}
\newcommand{\addTG}[1]{%
	\newglossaryentry{#1}{
		type=TG, 
		name={\syntaxTG#1}, 
		description={\nopostdesc}
	}%
}
\newcommand{\anchorTG}[1]{%
	\edef\theTG{\getTG{#1}}%
	\expandafter\addTG\expandafter{\theTG\expandafter}%
	\newglossaryentry{#1}{
		type=TG, 
		name={\textmd{#1}}, 
		parent={\theTG}, 
		sort={\theTG}, 
		description={\hfill}
	}%
	\glsadd{#1}%
}
\newcommand{\cc}[1]{#1\anchorTG{#1}}

\title{廣東新語·木語·諸山果\footnote{\texttt{https://ctext.org/library.pl?if=en\&file=24116\&page=115}}}
\author{屈大均}
\date{康熙二十六年}

\begin{document}

\maketitle
	
廣中山果,曰\cc{穀子}者,大如\cc{橄欖}而長,初亦苦澀,後甘,嫩者蜜漬之可食。曰\cc{餘甘子},樹高丈餘,葉如\cc{槐},子如\cc{川楝},白色,有文理,核作六稜,亦初苦澀後甘,行者以之生津,一名\cc{菴摩勒}。曰\cc{鬼目子},大如\cc{梅}、\cc{李},皮黃肉紅,味甚酸,人以爲蔬,以皮上有目,名\cc{鬼目},一曰\cc{麂木},麂者,鬼之譌也。曰\cc{山棗子},葉似\cc{梅},子如\cc{荔支},九月熟。曰\cc{卍果},果作卍字形,畫甚方正,蔕在字中不可見,生食香甘,一名\cc{蓬鬆子}。曰\cc{桾櫏子},樹似\cc{甘蕉},子如\cc{馬乳}而小,俗稱\cc{牛奶柹},亦曰\cc{牛乳子},廣人言乳曰奶,中有美漿若牛乳,故曰\cc{牛奶子},一名\cc{㮕棗}。曰\cc{五子},其狀如\cc{梨},有五核。曰\cc{千歲子},蔓生,子在根下,有綠鬚交加如織,一苞恒二百餘子,皮青黃,乾者殼肉相離,撼之有聲,似\cc{肉荳蔻}。曰\cc{秋風子},色褐,如梨而小,味酸澀,熟乃可食。曰\cc{金紐子},色紅黃,味甘,大如\cc{秋風子},俗歌云:「一雙\cc{金紐子},無計上羅衫。」曰\cc{青竹子},如\cc{桃}而圓,味酸,色黃。曰\cc{羊齒子},一曰\cc{羊矢},如\cc{石蓮}而小,色青味甘。曰\cc{不納子},實如圓\cc{棗},十月黃熟,味甜酸,蓋\cc{蘋果}之小者,粵中少\cc{蘋果}、\cc{花紅}二種,以\cc{不納子}代之。曰\cc{山葡萄},一名\cc{蘡薁},其莖細裊,花紫白,實比\cc{葡萄}而小,色赤,味酢,可爲酒,八九月熟。曰\cc{山韶子},類\cc{荔支}而鮮麗過之,微有小毫,一名\cc{毛荔支},亦曰\cc{毛珧子},肉薄而酸澀,著核不離,蓋\cc{荔支}之變者。曰\cc{臙脂子},子赤如臙脂,味甜酸,諺曰:「不采\cc{紅蓼}花,但采\cc{臙脂子};持以作朱顏,其餘入玉齒。」曰\cc{都捻子},樸樕叢生,花如\cc{芍藥}而小,春時開,有紅白二種,子如\cc{軟柿},外紫內赤,亦小,有四葉承之,每食必倒捻其蔕,故一名\cc{倒捻子},子汁可染若臙脂,花可爲酒,葉可麯,皮漬之得膠以代\cc{柿},蘇子瞻名曰\cc{海漆},非\cc{漆}而名爲\cc{漆},以其得乙木之液,凝而爲血,而可補人之血,與\cc{漆}同功,功逾\cc{青黏},故名,取子研濾爲膏,餌之又止腸滑,以其爲用甚眾,食治皆需,故又名\cc{都捻},產羅浮者高丈許,子尤美。曰\cc{黃皮果},狀如金彈,六月熟,其漿酸甘似\cc{葡萄},可消食順氣除暑熱,與\cc{荔支}並進,\cc{荔支}饜飫,以\cc{黃皮}解之,諺曰:「饑食\cc{荔支},飽食\cc{黃皮}。」有曰\cc{白蠟子}者,與相似,其味尤勝,諺曰:「\cc{黃皮}\cc{白蠟},甜酸相雜。」曰\cc{蘋婆果},一名\cc{林檎},樹極高,葉大而光潤,莢如\cc{皂角}而大,長二三寸,子生莢兩旁,或四或六,子老則莢迸開,內深紅,包子皮黑,肉黃,熟食味甘,蓋\cc{耎栗}也,相傳三藏法師從西域攜至,與\cc{訶梨勒}、\cc{菩提}雜植虞飜苑中,今遍粵中有之,梵語曰\cc{蘋婆},以其葉盛成叢,又曰\cc{叢林}。一種\cc{水樃子},與相似,多生水間,或謂\cc{林檎}爲雄,\cc{水樃}爲雌,粵歌云:「\cc{水樃}\cc{林檎}大小同,盤中不辩是雌雄。」

\section*{植物名稱四角號碼檢索表}

\setlength{\columnsep}{16pt}
\setlength{\columnseprule}{0.4pt}
\begin{multicols}{4}
	\renewcommand{\glossarysection}[2][]{}
	\renewcommand*{\glsgroupskip}{}
	\setlength{\glstreeindent}{7pt}
	\printglossary[type=TG,style=tree,title=]
\end{multicols}

\end{document}

%% End of file `tetrogonos-example.sty'
