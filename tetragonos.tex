\documentclass[a4paper]{ctexart}
% 20190115
% Author: Yuchang Yang < yang.yc.allium@gmail.com >
\usepackage{metalogo}
\usepackage{fancyvrb}
\usepackage{changepage}
\usepackage{color}
\definecolor{darkmiku}{RGB}{19, 149, 139}
\CTEXsetup[format={\Large\bfseries}]{section}

\title{tetragonos宏包\footnote{Github 地址:\texttt{https://github.com/Mikumikunisiteageru/tetragonos}}}
\author{杨宇昌\footnote{电子邮箱:\texttt{yang.yc.allium@gmail.com}}}
\date{2019年1月16日\qquad v1}

\begin{document}

\maketitle

\begingroup
	\CJKfontspec{IPAPMincho}
	\large
	\centerline{四角い地球を丸くしてる!}
	\hfill --- 初音ミク、\kern1ex TOKOTOKO(西沢さん\mbox{P})\hbox{}\par
	\vskip1em
\endgroup

\section{简介}
tetragonos 是一个用于查询汉字四角号码的\XeLaTeX{}宏包。\par
宏包名称取自希腊语,tetra者四也,gon者角也。\par
	
\section{用法}
tetragonos 宏包定义了宏 \verb|\getTG|。\texttt{\textbackslash getTG\{\textit{<string>}\}} 可展开成汉字字符串 \texttt{\textit{<string>}} 中第一个汉字的带附号的新版四角号码。举个简单的例子,\verb|\getTG{汉}| 与 \verb|\getTG{汉字}| 都会展开成“汉”字的四角号码,即 37140。\par
	
\section{四角索引范例}
四角号码常用来为汉字字词编写索引。文件 tetragonos-example.tex 中即示范了 tetragonos 宏包与 glossaries 宏包的联合应用。该文件从《广东新语》中摘录了一段文字,并为其中出现的所有植物名称建立了首字四角号码索引。该文件中的关键内容如下:
	
\begingroup
	\begin{adjustwidth}{3em}{3em}
	\footnotesize
	\begin{Verbatim}[numbers=left,numbersep=1ex,gobble=2,firstnumber=1,formatcom=\color{darkmiku}]
		% arara: clean: {files: [tetragonos-example.glsdefs, tetragonos-example.TG-*]}
		% arara: xelatex
		% arara: makeglossaries
		% arara: xelatex
		% arara: xelatex
	\end{Verbatim}
	指示 arara 运行时先清空历史索引条目,之后依次执行 \XeLaTeX{}、makeglossaries 等各程序。编译时直接输入 \verb|arara tetragonos-example| 即可。
	\begin{Verbatim}[numbers=left,numbersep=1ex,gobble=2,firstnumber=28,formatcom=\color{darkmiku}]
		\usepackage{tetragonos}
		\usepackage[nopostdot,nomain]{glossaries}
		\newglossary*{TG}{Tegragonos}
		\makeglossaries
	\end{Verbatim}
	引入 tetragonos 与 glossaries 宏包后,新建好索引,准备登记四角号码与汉字条目。
	\begin{Verbatim}[numbers=left,numbersep=1ex,gobble=2,firstnumber=32,formatcom=\color{darkmiku}]
		\def\syntaxTG#1#2#3#4#5{\textbf{#1#2#3#4\textsubscript{#5}}}
	\end{Verbatim}
	定义四角号码的格式,使其整体加粗,并令附号(即附角号码)作下标。
	\begin{Verbatim}[numbers=left,numbersep=1ex,gobble=2,firstnumber=33,tabsize=4,formatcom=\color{darkmiku}]
		\newcommand{\addTG}[1]{%
			\newglossaryentry{#1}{
				type=TG, 
				name={\syntaxTG#1}, 
				description={\nopostdesc}
			}%
		}
	\end{Verbatim}
	定义宏 \verb|\addTG|,建立四角号码条目,在索引中位于第一级。
	\begin{Verbatim}[numbers=left,numbersep=1ex,gobble=2,firstnumber=40,tabsize=4,formatcom=\color{darkmiku}]
		\newcommand{\anchorTG}[1]{%
			\edef\theTG{\getTG{#1}}%
			\expandafter\addTG\expandafter{\theTG\expandafter}%
			\newglossaryentry{#1}{
				type=TG, 
				name={\textmd{#1}}, 
				parent={\theTG}, 
				sort={\theTG}, 
				description={\hfill}
			}%
			\glsadd{#1}%
		}
	\end{Verbatim}
	定义宏 \verb|\anchorTG|,建立汉字(词)条目,居于第二级,是其首字四角号码对应的条目的子条目。\verb|\anchorTG| 只插入锚点,登记汉字(词)条目与当前页码,不在正文中产生可视效应。这是照顾一些在正文与索引中有不同字(词)形的条目的需要,毕竟在 \TeX{} 里合总比分容易。glossaries 宏包中的 \verb|\newglossaryentry| 命令会自动忽略登记过的条目,因此不必特意对条目判重。\verb|\expandafter| 强迫 \verb|\theTG| 在送入 \verb|\addTG| 前完全展平成四角号码。
	\begin{Verbatim}[numbers=left,numbersep=1ex,gobble=2,firstnumber=52,formatcom=\color{darkmiku}]
		\newcommand{\cc}[1]{#1\anchorTG{#1}}
	\end{Verbatim}
	\verb|\cc| 则定义为显示完汉字(词)后马上把它原样加入索引的命令。
	\begin{Verbatim}[numbers=left,numbersep=1ex,gobble=2,firstnumber=68,tabsize=4,formatcom=\color{darkmiku}]
		\begin{multicols}{4}
			\renewcommand{\glossarysection}[2][]{}
			\renewcommand*{\glsgroupskip}{}
			\setlength{\glstreeindent}{7pt}
			\printglossary[type=TG,style=tree,title=]
		\end{multicols}
	\end{Verbatim}
	四角号码索引分成多个纵栏,采用 tree 样式,第一级的四角号码均无缩进,第二级的汉字(词)排于各自所属的四角号码之后,统一缩进值可根据需要自由调整。
	\end{adjustwidth}
\endgroup

\section{代码实现}

宏包主文件 tetragonos.sty 的全部内容如下:

\begingroup
	\begin{adjustwidth}{3em}{3em}
	\footnotesize
	\begin{Verbatim}[numbers=left,numbersep=1ex,gobble=2,formatcom=\color{darkmiku}]
		%% tetragonos.sty
		%% Copyright 2019 Yuchang Yang < yang.yc.allium@gmail.com >
		%
		% This work may be distributed and/or modified under the
		% conditions of the LaTeX Project Public License, either version 1.3
		% of this license or (at your option) any later version.
		% The latest version of this license is in
		%   http://www.latex-project.org/lppl.txt
		% and version 1.3 or later is part of all distributions of LaTeX
		% version 2005/12/01 or later.
		%
		% This work has the LPPL maintenance status `maintained'.
		% 
		% The Current Maintainer of this work is Yuchang Yang.
		%
		% This work consists of the files tetragonos.sty and tetragonos-database.def
		% and the associated example file tetragonos-example.tex.
		%%
		\NeedsTeXFormat{LaTeX2e}
		\ProvidesPackage{tetragonos}[2019/01/15 v1 package tetragonos]
		\newcommand{\saveTG}[2]{\expandafter\gdef\csname TG-#1\endcsname{#2}}
		\newcommand{\loadTG}[1]{\csname TG-#1\endcsname}
	\end{Verbatim}
	宏 \verb|\saveTG| 与 \verb|\loadTG| 实际上实现了一个索引为汉字字符的数组的写入与读出。
	\begin{Verbatim}[numbers=left,numbersep=1ex,gobble=2,firstnumber=last,formatcom=\color{darkmiku}]
		\makeatletter
		\newcommand{\get@first@char}[1]{\loadTG{#1}\iffalse}
		\newcommand{\getTG}[1]{\get@first@char#1\fi}
		\makeatother
	\end{Verbatim}
	宏 \verb|\getTG| 利用 \TeX{} 的宏机制配合 \verb|\get@first@char|,实现了抓取参数中的第一个字符并查询其四角号码的功能。
	\begin{Verbatim}[numbers=left,numbersep=1ex,gobble=2,firstnumber=last,formatcom=\color{darkmiku}]
		\input{tetragonos-database.tex}
		%%
		%% End of file `tetrogonos.sty'
	\end{Verbatim}
	\end{adjustwidth}
\endgroup

汉字到四角号码的映射储存在另外的文件 tetragonos-example.def 中。其正文开头数行如下:

\begingroup
	\begin{adjustwidth}{3em}{3em}
	\footnotesize
	\begin{Verbatim}[numbers=left,numbersep=1ex,gobble=2,formatcom=\color{darkmiku},firstnumber=22]
		\ProvidesFile{tetragonos-database.def}[2019/01/14 v1 tetragonos database]
		\saveTG{亠}{00000}
		\saveTG{弯}{00027}
		\saveTG{亪}{00037}
		\saveTG{疒}{00100}
		\saveTG{韲}{00101}
	\end{Verbatim}
	\end{adjustwidth}
\endgroup

\end{document}
